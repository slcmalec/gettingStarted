\documentclass{ximera}
\usepackage[colorlinks=true,urlcolor=blue]{hyperref}
\title{Setting up the repository}
\begin{document}
\begin{abstract}
Instructions for setting up a repository containing course materials.
\end{abstract}
\maketitle
\begin{enumerate}
\item Create a directory for your course files
and change to that directory.
In this example, we will create a directory called \verb!sampleCourse!.
\begin{center}
\begin{verbatim}
mkdir sampleCourse
cd sampleCourse 
\end{verbatim}
\end{center}
\item A Ximera course consists of a directory containing
minimally a text file called \verb!course.xim!. This file should contain
the name of the course, a description of the course,
and the names of all the \LaTeX\ activity files 
comprising the course, in the order
that they should be presented to students.


\begin{verbatim}
---
name: Getting Started with Ximera
description: This is a Ximera activity explaining how to get started
with Ximera for course instructors.
---

example/example
\end{verbatim}

\begin{warning}
Note that the three dashes encapsulating the name and description
are required, as is the blank line following the second set of dashes!
\end{warning}

We recommend placing each
activity in its own directory, which should have the same name.
This facilitates the
sharing of activities between collaborators and makes reusing existing
activities much easier.
%Later in this course, we will see examples of
%how to borrow existing activities from other courses
%rather than starting from scratch. 
In the example above
there is one activity file \verb!example.tex!
written without the extension \verb!.tex! We will
create this file in the next step.
Generally courses should have more activities than this
one, with each activity name in \verb!course.xim!
indented to reflect its position in the course hierarchy.

\item Next we create a directory
called \verb!example! containing a file called \verb!example.tex!.
\begin{verbatim}
mkdir example
cd example
touch example.tex
\end{verbatim}

\item The activity contained in \verb!example.tex!
should be in the document class \verb!ximera!
and contain the title of the activity
and an abstract. For example,
\begin{verbatim}
\documentclass{ximera}
\title{The First Activity}
\begin{document}
\begin{abstract}
This activity deals with \verb!Ximera! activities
\end{abstract}
\maketitle
\end{document}
\end{verbatim}
The activity at this stage contains no content.

\item On \href{http://github.org}{\tt github.org},
create a repository with the same name as your directory
by clicking the \verb!+! by your account name, as shown
below.

\begin{image}
\includegraphics[scale=.3]{RepoInit.png}
\end{image}

\item Follow the instructions in \href{http://github.org}{\tt github.org} to set up
the repository, accepting all default settings.
\href{http://github.org}{\tt github.org}
will respond with additional directions, which you should execute.
\begin{verbatim}
touch README.md
git init
git add README.md
git commit -m "first commit"
git remote add origin https://github.com/marcusjbishop/example.git
git push -u origin master
\end{verbatim}
Run these commands as directed. This initializes the repository on 
\href{http://github.org}{\\tt github.org}.

\item Optionally write a description
of the course in \verb!README.md!

\item Click on the \verb!settings! button on the
\href{http://github.org}{\tt github.org} repository page
and then on \verb!Webhooks & Services!
\begin{center}
\begin{image}
\includegraphics[scale=.3]{Webhook.png}
\end{image}
\end{center}
Now click the \verb!Add webhook! button.
Put \verb!http://ximera.osu.edu/github!
into the \verb!Payload URL! field and
\verb!8mi0tsrje9n3asPu86XC198G1XSdZj!
into the \verb!Secret! field.

\item In order to have our course listed on
\href{http://ximera.osu.edu/course}{\tt ximera.osu.edu/course}
we need to \verb!push! the repository.
However, since \verb!git! will not allow a \verb!push!
without any changes to the repository, this is a good point
to add content in the form of a simple exercise within the 
\verb!example.tex! file.
\begin{verbatim}
\documentclass{ximera}
\title{The First Activity}
\begin{document}
\begin{abstract}
This activity deals with \verb!Ximeara! activities.
\end{abstract}
\maketitle
This activity is about creative work.
\begin{exercise}
  Choose the best place to work on mathematics.
  \begin{multiple-choice}
    \choice{At the library}
    \choice[correct]{At the caf\'e}
    \choice{In your office}
  \end{multiple-choice}
\end{exercise}
\end{document}
\end{verbatim}
See the following activity for more information on creating
exercises.

\item Change to the root directory of your repository
and execute the following commands.
\begin{verbatim}
git commit -am "Added an exercise"
git push
\end{verbatim}

\item If everything went well, you should find your course listed at 
\href{http://ximera.osu.edu/course}{\tt ximera.osu.edu/course}.
If not, see the troubleshooting activity.
\end{enumerate}
\end{document}
