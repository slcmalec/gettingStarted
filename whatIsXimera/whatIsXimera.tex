\documentclass{ximera}
\usepackage[colorlinks=true,linkcolor=blue]{hyperref}
\title{What is Ximera?}
\begin{document}
\begin{abstract}
An introduction to the Ximera system.
\end{abstract}
\maketitle

\href{http://ximera.osu.edu}{\sf Ximera}
is an open-source software project that
seeks to help course instructors create learning materials
for their students simultaneously as PDF files and as interactive
web pages. Our strategy to achieve this goal is to separate
content from deployment.

An author writes content in the form of a \LaTeX\ document.
This produces a PDF handout that can be distributed to students.
Next the author delivers this same \LaTeX\ file to
\href{http://github.com}{\tt github.com},
a free web-based service providing a number of features to developers.
In turn \href{http://github.com}{\tt github.com} delivers
the file to the \href{http://ximera.osu.edu}{\sf Ximera}
interpreter, which responds by posting the file on the web.
The web page has essentially the same content as the handout.
However, the web page typically has interactive features
not possible in the handout due to the physical properties of paper.
For example, the web page might pose a question
that if answered incorrectly, offers hints or further questions
to the student.
This process is illustrated in the figure below.

\begin{image}
\includegraphics[scale=.25]{XimeraGraphic.png}
\end{image}

\subsection{Benefits of Ximera}
One benefit of \href{http://ximera.osu.edu}{\sf Ximera}
is that it provides an easy way to produce interactive online materials.
Another benefit is that many educators, particularly mathematicians,
are already quite familiar with \LaTeX\ and
even find it quite easy to use.
And because the \TeX\ language is extremely static in comparison with
other programming and markup languages, authors can expect
the materials they create to be usable in some
form for the foreseeable future.

\subsection{Differences between Ximera and other
Learning Management Systems}
A Learning Management System
(LMS) such as Blackboard or Moodle
provides students with a webpage from
which they can navigate to the webpages of
each course in which they are registered, all of
which formatted and laid out in exactly the same way.
Because of the possibility of sharing grades
with students, users have accounts on the LMS,
which are typically set up by
the college or University, as are the course
web pages themselves.

By far, the most common way to use an LMS in our
experience is to make announcements or distribute PDF files to students.
While this could be accomplished through email
or a simple web page, the LMS, being
largely set up by the college or University, provides
and even easier way to distribute files and communicate with students.
It also organizes students' course materials in a single location.

Notwithstanding, another feature provided by an LMS is assessment.
Instructors set up quizzes or homework assignments,
which consist of sequences of questions of various types.
Typically each question has a unique answer, allowing
the LMS to score the assignment and record grades in a
gradebook, another feature of an LMS.
It should be born in mind that composing questions for
an LMS, particularly multiple-choice questions, creates a lot of work
for instructors. For this reason textbook publishers
are increasingly bundling their books with accounts
on the publisher's proprietary LMS, from which an
instructor can select precomposed exercises for students
to complete.

In contrast \href{http://ximera.osu.edu}{\sf Ximera}
provides a more flexible environment for creating course materials.
While an activity on an LSM consists solely of a sequence
of problems, a \href{http://ximera.osu.edu}{\sf Ximera}
activity can contain whatever content a \LaTeX\ document
can contain, and in addition, can include interactive questions
of various types.

Whereas an LMS typically only supplements a traditional course,
an entire course can be delivered through
\href{http://ximera.osu.edu}{\sf Ximera}.
For example, a course could consist of a sequence
of \href{http://ximera.osu.edu}{\sf Ximera}
activities, each of which primarily composed of text
for students to read, with occasional 
questions sprinkled in to confirm and deepen students'
understanding of the material appearing just before the question.
More substantial questions could appear at the end of an activity.


\end{document}
